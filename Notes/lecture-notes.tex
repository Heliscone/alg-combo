\documentclass[11pt]{scrartcl}
\usepackage[bagel]{evan}

\title{Algebraic Combinatorics}
\author{Evan Lim}											
\date{Spring 2024}
\definecolor{palegreen}{rgb}{0.6, 0.98, 0.6}
\begin{document}
\maketitle
We're not studying this in silos, because of the great molasses flood where silos were covered in molasses. And why was that? Because people kept putting their mathematics in silos.
\section{Walks in Graphs}
Let $G=(V,E)$ be an (undirected) graph where $E$ is a multi-subset of $\binom{V}{2}$, that is, $E\subseteq\binom{V}{2}$.

Let there be $n$ vertices ($|V|=n$). The adjacency matrix $A=A(G)$ is an $n\times n$ matrix where $a_{ij}$ is the number of edges incident on $v_i$ and $v_j$.

A walk of length $l$ in $G$ going from $u$ to $v$ is a sequence of vertices and edges \[u=v_1,e_1,v_2,e_2,\cdots,e_{l},v_{l+1}=v.\] Note distinction with path/trail.

\begin{theorem}
    For any positive integer $l$, the $(A(G))^l_{ij}$ is the number of walks of length $l$ from $v_i$ to $v_j$.
\end{theorem}
\begin{proof}
    This facts comes from the dot product computation of the $ij$th entry of the adjacency matrix power: \[(A(G)^l)_{ij}=\sum_{\text{``walks"}} a_{ii_1}a_{i_1i_2}\cdots a_{i_{l-1}j}\]
\end{proof}
\begin{remark}
    The total number of closed 1-walks is simply $\text{tr}(A)=\sum\lambda_i$.
\end{remark}
Note that $A$ is symmetric, so $A^T=A$ and we can write the diagonalization and trivially get \[A^l=U\text{diag}(\lambda_i^l)U^{-1},\] so it follows immediately that \[\text{tr}(A^l)=\sum\lambda_i^l.\]


\end{document}
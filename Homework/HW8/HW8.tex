\documentclass[11pt,letterpaper]{article}
\usepackage[lightbagel]{evan}

\title{Algebraic Combinatorics - HW8}
\author{Evan/Dallin/Xander/Sawyer}
\date{4/16/2024}
\begin{document}
\maketitle
\begin{quest}[\textcolor{red}{Saving electricity through Linear Algebra and Graph Theory}]
    \begin{enumerate}[(i)]
        \item Consider the vector space $V=\mathbb{F}_2^n$ (over $\mathbb{F}_2)$. Let $A_{n\times n}$ be a symmetric matrix over $\mathbb{F}_2$. Consider the diagonal of $A$, as a column vector $\overrightarrow{d}$. Prove that 
        \[\vec{d}\in Col(A)\]
        (\textit{Hint}: First show that $\vec{v}^{T}A\vec{v}=\vec{d}^{T}\vec{v}$ for all $\vec{v}\in V$. Then show the required result by contradiction.)
        \item Given a graph $G$, show (using $(i)$) that $V(G)$ can be partitioned into $V_1$ and $V_2$ such that $G[V_1]$ is an even graph (that is, all degrees are even) and for all vertices $v\in V_2$, $|N(v)\cap V_1|$ is odd.
        \item Assume that there is a bulb and a button at each vertex of a graph $G$. The connections are made such that pushing the button at a vertex once, will change the status of the bulb and its neighbors. Initially all bulbs are on. Show (using $(ii)$) that one can push some buttons and turn all the bulbs off. Does this remind you of a game that you may have played as a kid?
    \end{enumerate}
\end{quest}
\begin{proof}
    \begin{enumerate}[(i)]
        \item By matrix multiplication, \[v^TAv=\sum_{a_{ij}\in A}a_{ij}v_iv_j=\sum a_{ii}v_i^2+0=d^Tv.\]
        Note that $(Ax)^Ty=(Ay)^Tx$ by dot product properties. 
        
        Thus, the image of $A$ is the orthogonal complement of the kernel of $A^T=A$.

        If $Av=0$, then $d^Tv=0$, so $d$ is orthogonal to the Kernel of $A$, whence $d\in \text{Col}(A)$.

        \item 
    \end{enumerate}

\end{proof}
\begin{quest}[\textcolor{red}{Cycle Space and Bond Space in Graphs + some applications}]
    Given an undirected connected graph $G$ with $n$ vertices and $m$ edges, let each subset of the edge set $E(G)$ be represented by its characteristic binary vector. Let 
    \begin{align*}
        \mathcal{C}(G)&=\{\text{all even subgraphs of } G \}\\
        \mathcal{B}(G)&=\{\text{all minimal edge-cuts of } G\}\\
        M(G)&= \text{incidence matrix of } G
    \end{align*}
    \begin{enumerate}[(i)]
        \item Show that both $\mathcal{C}(G)$ and $\mathcal{B}(G)$ are subspaces of $\mathbb{F}_2^m$ (over $\mathbb{F}_2)$.
        \item Show that the \textit{stars} at any $n-1$ vertices of $G$ are linearly independent and forms a basis in $\mathcal{B}(G)$, thus \[\mathcal{B}(G)=\text{Row}(M(G))\quad\text{and}\quad \dim(\mathcal{B}(G))=n-1.\]
        \item Given any spanning tree $T$ of $G$, show that the \textit{fundamental cycles} (as described in class) are linearly independent in $\mathcal{C}(G)$. Then show that 
        \[\mathcal{C}(G)=(\mathcal{B}(G))^{\perp}\quad\mathcal{C}(G)=\text{Nul}(M(G))\quad \dim\big(\mathcal{C}(G)\big)=m-n+1\]
        \item Use the above to show that:
        \begin{enumerate}
            \item A graph is bipartite $\iff$ every circuit is of even length. (\textit{Hint}: Show that $\overrightarrow{j}=(1,1,\cdots, 1)\in \mathcal{B}(G)$)
            \item Show that for any graph $G$, $E(G)$ can be partitioned into an even graph and an edge-cut.
            \item Use $(b)$ to give a new proof of $\# 1(ii)$ (and hence $\# 1(iii)$)
        \end{enumerate}
    \end{enumerate}
\end{quest}
\begin{proof}
    \begin{enumerate}[(i)]
    \item Clearly, over $\F_2$, scalar multiplication is no issue, as scalar multiples of an edgeset are either the empty set or that same edgeset. We must show that the ``addition'' of two graphs in $\mathcal{C}$ or $\mathcal{B}$, that is, their symmetric difference, remains in $\mathcal{C}$ or $\mathcal{B}$, respectively. $\mathcal{C}$ is clear; given $C_1,C_2\in\mathcal{C}$, consider the degree of a vertez in $C_1+C_2$. Clearly, the sum of two even degrees is even, and is unchanged over $\F_2$. For $\mathcal{B}$, consider two minimal edge cuts $B_1$ and $B_2$, which yield vertex partitions $v_1, w_1$ and $v_2, w_2$, respectively.

    We can draw a picture that represents the four potential $v_i$ and $w_i$ that a vertex can be in, as well as edges between these partitions:
    \begin{center}
        \begin{asy}
            size(8cm);
            pair A=3.2*dir(180-45),B=3.2*dir(45),C=3.2*dir(-45),D=3.2*dir(-135);
            pair d1=(1,2),d2=(-1,2),d3=(-1,-2),d4=(1,-2);
            draw((A+d1)--(A+d2)--(A+d3)--(A+d4)--cycle);
            draw((B+d1)--(B+d2)--(B+d3)--(B+d4)--cycle);
            draw((C+d1)--(C+d2)--(C+d3)--(C+d4)--cycle);
            draw((D+d1)--(D+d2)--(D+d3)--(D+d4)--cycle);
            draw((A+1.2*d1)--(A+1.2*d2)--(D+1.2*d3)--(D+1.2*d4)--cycle,blue+0.05cm+dashed);
            draw((B+1.2*d1)--(B+1.2*d2)--(C+1.2*d3)--(C+1.2*d4)--cycle,green+0.05cm+dashed);
            draw((A+1.1*d3)--(A+1.1*d2)--(B+1.1*d1)--(B+1.1*d4)--cycle,red+0.05cm+dotted);
            draw((D+1.1*d3)--(D+1.1*d2)--(C+1.1*d1)--(C+1.1*d4)--cycle,orange+0.05cm+dotted);
            label("$v_1$",(A.x+d2.x-1,0),blue);
            label("$w_1$",(B.x+d1.x+1,0),green);
            label("$v_2$",(0,A.y+d2.y+1),red);
            label("$w_2$",(0,D.y+d3.y-1),orange);
            dot("$A$",A,N);
            dot("$B$",B,N);
            dot("$C$",C,S);
            dot("$D$",D,S);
            draw(A--D,blue);
            draw(B--C,green);
            draw(A--C);
            draw(B--D);
            draw(C--D,orange);
            draw(A--B,red);
        \end{asy}
    \end{center}
    $XY$ represents all edges between vertices in $X$ and $Y$. $A$ represents all edges within the vertex set defined by $v_1\cap v_2$. Note that \[B_1=AB\cup CD\cup AC\cup DB\] and \[B_2=AD\cup BC\cup AC\cup BD,\] so the symmetric difference of $B_1$ and $B_2$ is simply \[AB\cup BC\cup CD\cup DA,\] which is a minimal edge cut that yields partitions \[z_1=(v_1\cap v_2)\cup(w_1\cap w_2)\] and \[z_2=(v_2\cap w_1)\cup(v_1\cap w_2).\]
    \item Let $s_i$ be the star at vertex $v_i$. Select $n-1$ stars, WLOG $s_1$ through $s_{n-1}$. Clearly, any $s_i\in\mathcal{B}$, with vertex partitions $v_i$ and the non-$v_i$ vertices.
    
    To show that the $n-1$ stars are linearly independent, we show that if $\sum_{i=1}^{n-1}a_is_i=0$, then all $a_i=0$. We can do this by considering an edge $v_nv_i$, which is nonzero only in $s_n$ and $s_i$, so $a_i=0$ for all $a_i$.
    
    Note that $s_1+s_2+\cdots+s_n=0$, since an edge from $v_i$ to $v_j$ is represented once in $s_i$ and $s_j$, and we are over $\F_2$. Thus, $s_n=s_1+\cdots+s_{n-1}$.
    
    Consider a bond $b\in\mathcal{B}$ that separates the vertices into $v_{a_k}$ and $v_{b_k}$ where $a_k$ and $b_k$ are disjoint and union to $1,\dots,n$. Then we may write \[b=\sum_k s_{a_k},\] and this representation is unique over $\F_2$. Thus, $\mathcal{B}=\text{Row}(M(G))$ and $\dim(\mathcal{B})=n-1$.
    
    The star for vertex $k$ represents the transpose of the $k$th row of $M(G)$, so $\mathcal{B}(G)=\text{Row}(M(G))$. The dimension of $\mathcal{B}$ is the number of linearly independent stars (elements in the basis), which is $n-1$.

    \item The fundamental cycles are linearly independent and form a basis by a similar argument to that of the bonds: an edge $e$ not on $T$ is included only in the fundamental cycle generated by itself.

    Take some $c\in\mathcal{C}$ and $b\in\mathcal{B}$. The bond represented by $b$ splits, by definition, the graph $G$ into two partitions. If $b$ and $c$ share some edges, they must share an even number of edges, by the minimality of $b$. If $b$ and $c$ shared an odd number of edges, there would be some edge we could remove and still have an edge cut. For example, adding one edge on the cycle to the bond still leaves the vertices along the cycle connected. Thus, $c$ and $b$ both have ones only in an even number of coordinates, whence $c\cdot b=0$, and thereby every cycle $c$ is orthogonal to every bond $b$. Therefore, $\mathcal{C}(G)=(\mathcal{B}(G))^{\perp}$, as desired.
    
    Then, $\mathcal{C}(G)=(\mathrm{Row}(M(G)))^{\perp}=\mathrm{Nul}(M(G))$. The dimension theorem gives $\dim\mathcal{C}=m-\dim\mathcal{B}=m-n+1$.

    \item \begin{enumerate}[(a)]
        \item $\vec j\in\mathcal{B}$ means that the set of all edges is a minimal cut. If we consider the vertex partitions $v,v'$ cut by $\vec j$, it is clear that there are no edges within $v$ nor edges within $v'$, so $G$ is bipartite. The opposite direction is obvious. Therefore, it suffices to show that $\vec j\in\mathcal{B}$ if and only if every circuit is of even length.
        
        If $\vec j\in\mathcal{B}$, by the orthogonality of $\mathcal{C}$ and $\mathcal{B}$, $\vec j\cdot c=0$ for all $c\in\mathcal{C}$. Since we are working over $\F_2$, this means that $c$ has an even number of ones, so all circuits are of even length.

        If all circuits are of even length, an identical argument gives $\vec j\cdot c=0$ for all $c\in\mathcal{C}$, so $\vec j\in\mathcal{B}$.
        \item By the orthogonality of $\mathcal{C}$ and $\mathcal{B}$, we can partition any graph $G$'s edge set into a cycle and an edge-cut. Cycles are even graphs, so we can partition $E(G)$ into an even graph and an edge-cut.
        \item It suffices to show that for such a partition into a cycle and an edge-cut, for any vertex $v$ in the edge cut, the intersection between the neighborhood of $v$ and the cycle is odd. This is clear, as the edge cut separates the graph into two partitions, and the cycle is a cycle, so the intersection is odd.
        
        However, it is unclear whether/how we can make the jumps between reasoning with edges and vertices.
    \end{enumerate}
    \end{enumerate}
    
\end{proof}
\end{document}
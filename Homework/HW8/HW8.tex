\documentclass[11pt,letterpaper]{article}
\usepackage[lightbagel]{evan}

\title{Algebraic Combinatorics - HW8}
\author{Evan/Dallin/Xander/Sawyer}
\date{4/16/2024}
\begin{document}
\maketitle
\begin{quest}[\textcolor{red}{Saving electricity through Linear Algebra and Graph Theory}]
    \begin{enumerate}[(i)]
        \item Consider the vector space $V=\mathbb{F}_2^n$ (over $\mathbb{F}_2)$. Let $A_{n\times n}$ be a symmetric matrix over $\mathbb{F}_2$. Consider the diagonal of $A$, as a column vector $\overrightarrow{d}$. Prove that 
        \[\vec{d}\in Col(A)\]
        (\textit{Hint}: First show that $\vec{v}^{T}A\vec{v}=\vec{d}^{T}\vec{v}$ for all $\vec{v}\in V$. Then show the required result by contradiction.)
        \item Given a graph $G$, show (using $(i)$) that $V(G)$ can be partitioned into $V_1$ and $V_2$ such that $G[V_1]$ is an even graph (that is, all degrees are even) and for all vertices $v\in V_2$, $|N(v)\cap V_1|$ is odd.
        \item Assume that there is a bulb and a button at each vertex of a graph $G$. The connections are made such that pushing the button at a vertex once, will change the status of the bulb and its neighbors. Initially all bulbs are on. Show (using $(ii)$) that one can push some buttons and turn all the bulbs off. Does this remind you of a game that you may have played as a kid?
    \end{enumerate}
\end{quest}
\begin{proof}
    
\end{proof}
\begin{quest}[\textcolor{red}{Cycle Space and Bond Space in Graphs + some applications}]
    Given an undirected connected graph $G$ with $n$ vertices and $m$ edges, let each subset of the edge set $E(G)$ be represented by its characteristic binary vector. Let 
    \begin{align*}
        \mathcal{C}(G)&=\{\text{all even subgraphs of } G \}\\
        \mathcal{B}(G)&=\{\text{all minimal edge-cuts of } G\}\\
        M(G)&= \text{incidence matrix of } G
    \end{align*}
    \begin{enumerate}[(i)]
        \item Show that both $\mathcal{C}(G)$ and $\mathcal{B}(G)$ are subspaces of $\mathbb{F}_2^m$ (over $\mathbb{F}_2)$.
        \item Show that the \textit{stars} at any $n-1$ vertices of $G$ are linearly independent and forms a basis in $\mathcal{B}(G)$, thus \[\mathcal{B}(G)=\text{Row}(M(G))\quad\text{and}\quad \dim(\mathcal{B}(G))=n-1.\]
        \item Given any spanning tree $T$ of $G$, show that the \textit{fundamental cycles} (as described in class) are linearly independent in $\mathcal{C}(G)$. Then show that 
        \[\mathcal{C}(G)=(\mathcal{B}(G))^{\perp}\quad\mathcal{C}(G)=\text{Nul}(M(G))\quad \dim\big(\mathcal{C}(G)\big)=m-n+1\]
        \item Use the above to show that:
        \begin{enumerate}
            \item A graph is bipartite $\iff$ every circuit is of even length. (\textit{Hint}: Show that $\overrightarrow{j}=(1,1,\cdots, 1)\in \mathcal{B}(G)$)
            \item Show that for any graph $G$, $E(G)$ can be partitioned into an even graph and an edge-cut.
            \item Use $(b)$ to give a new proof of $\# 1(ii)$ (and hence $\# 1(iii)$)
        \end{enumerate}
    \end{enumerate}
\end{quest}
\end{document}
\documentclass[11pt,letterpaper]{article}
\usepackage[lightbagel]{evan}

\title{Algebraic Combinatorics - HW8}
\author{Evan/Dallin/Xander/Sawyer}
\date{4/16/2024}
\begin{document}
\maketitle
\begin{quest}[\textcolor{red}{Saving electricity through Linear Algebra and Graph Theory}]
    \begin{enumerate}[(i)]
        \item Consider the vector space $V=\mathbb{F}_2^n$ (over $\mathbb{F}_2)$. Let $A_{n\times n}$ be a symmetric matrix over $\mathbb{F}_2$. Consider the diagonal of $A$, as a column vector $\overrightarrow{d}$. Prove that 
        \[\vec{d}\in Col(A)\]
        (\textit{Hint}: First show that $\vec{v}^{T}A\vec{v}=\vec{d}^{T}\vec{v}$ for all $\vec{v}\in V$. Then show the required result by contradiction.)
        \item Given a graph $G$, show (using $(i)$) that $V(G)$ can be partitioned into $V_1$ and $V_2$ such that $G[V_1]$ is an even graph (that is, all degrees are even) and for axll vertices $v\in V_2$, $|N(v)\cap V_1|$ is odd.
        \item Assume that there is a bulb and a button at each vertex of a graph $G$. The connections are made such that pushing the button at a vertex once, will change the status of the bulb and its neighbors. Initially all bulbs are on. Show (using $(ii)$) that one can push some buttons and turn all the bulbs off. Does this remind you of a game that you may have played as a kid?
    \end{enumerate}
\end{quest}
\begin{proof}
    By matrix multiplication, \[v^TAv=\sum_{a_{ij}\in A}a_{ij}v_iv_j=\sum a_{ii}v_i^2+0=d^Tv.\]
    Note that $(Ax)^Ty=(Ay)^Tx$ by dot product properties. 
    
    Thus, the image of $A$ is the orthogonal complement of the kernel of $A^T=A$.

    If $Av=0$, then $d^Tv=0$, so $d$ is orthogonal to the Kernel of $A$, whence $d\in \text{Col}(A)$.


\end{proof}
\begin{quest}[\textcolor{red}{Cycle Space and Bond Space in Graphs + some applications}]
    Given an undirected connected graph $G$ with $n$ vertices and $m$ edges, let each subset of the edge set $E(G)$ be represented by its characteristic binary vector. Let 
    \begin{align*}
        \mathcal{C}(G)&=\{\text{all even subgraphs of } G \}\\
        \mathcal{B}(G)&=\{\text{all minimal edge-cuts of } G\}\\
        M(G)&= \text{incidence matrix of } G
    \end{align*}
    \begin{enumerate}[(i)]
        \item Show that both $\mathcal{C}(G)$ and $\mathcal{B}(G)$ are subspaces of $\mathbb{F}_2^m$ (over $\mathbb{F}_2)$.
        \item Show that the \textit{stars} at any $n-1$ vertices of $G$ are linearly independent and forms a basis in $\mathcal{B}(G)$, thus \[\mathcal{B}(G)=\text{Row}(M(G))\quad\text{and}\quad \dim(\mathcal{B}(G))=n-1.\]
        \item Given any spanning tree $T$ of $G$, show that the \textit{fundamental cycles} (as described in class) are linearly independent in $\mathcal{C}(G)$. Then show that 
        \[\mathcal{C}(G)=(\mathcal{B}(G))^{\perp}\quad\mathcal{C}(G)=\text{Nul}(M(G))\quad \dim\big(\mathcal{C}(G)\big)=m-n+1\]
        \item Use the above to show that:
        \begin{enumerate}
            \item A graph is bipartite $\iff$ every circuit is of even length. (\textit{Hint}: Show that $\overrightarrow{j}=(1,1,\cdots, 1)\in \mathcal{B}(G)$)
            \item Show that for any graph $G$, $E(G)$ can be partitioned into an even graph and an edge-cut.
            \item Use $(b)$ to give a new proof of $\# 1(ii)$ (and hence $\# 1(iii)$)
        \end{enumerate}
    \end{enumerate}
\end{quest}
\begin{proof}
    Clearly, over $\F_2$, scalar multiplication is no issue. We must show that the ``addition'' of two graphs in $\mathcal{C}$ or $\mathcal{B}$, that is, their symmetric difference, remains in $\mathcal{C}$ or $\mathcal{B}$, respectively. $\mathcal{C}$ is clear; given $C_1,C_2\in\mathcal{C}$, consider the degree of a vertez in $C_1+C_2$. Clearly, the sum of two even degrees is even, and is unchanged over $\F_2$. For $\mathcal{B}$, consider two minimal edge cuts $B_1$ and $B_2$, yielding partitions $v_1, v_1'$ and $v_2, v_2'$, respectively. $B_1$ consists of all edges between $v_1$ and $v_1'$ and likewise for $B_2$, $v_2$, and $v_2'$.

    The only edges that remain are those in $B_1\setminus B_2$ or $B_2\setminus B_1$. These connect vertices in $v_1\setminus v_2$ to $v_1'\setminus v_2'$ or $v_2\setminus v_1$ to $v_2'\setminus v_1'$.
    FINISH THIS ARGUMENT.
    
    Let $s_v$ be the star at vertex $v$. All $s_i$ are evidently in $\mathcal{B}$, and if we have a minimal edge-cut $b$ between $v_{i_1},\dots,v_{i_k}$ and $v_{j_1},\dots,v_{j_l}$, then note that $b=\sum_{c=1}^k s_{v_{i_c}}$, remembering that we are over $\FF_2$. (Another consequence of working over this field is that the sum is identical to $\sum_{c=1}^ls_{v_{j_c}}$.) We now need only write $s_1+s_2+\cdots+s_{n-1}=s_n$, which we can do by edge-counting: for an edge between $v_n$ and $v_m$, $s_m$ includes this edge, and no other $s_i$ does. Thus, the edge's coordinate remains nonzero in the partial sum of the $s_i$. Similarly, for an edge between $v_m$ and $v_o$ with $m,o\neq n$, $s_m$ and $s_o$ both include this edge, so it is not included in the partial sum. Thus, the only remaining edges in the partial sum are those connected to $v_n$, whence the sum is $s_n$, so we can write the $n$th star as a sum of the others. A quick examination of the $n-1$ other stars shows that they are linearly independent, by consider the coordinate representing an edge between $v_i$ and $v_n$: it is nonzero only in $s_i$, so if $\sum_{i}a_is_i=0$, all $a_i=0$.

    The star for vertex $k$ represents the transpose of the $k$th row of $M(G)$, so $\mathcal{B}(G)=\text{Row}(M(G))$. The dimension of $\mathcal{B}$ is the number of linearly independent stars (elements in the basis), which is $n-1$.

    
\end{proof}
\end{document}
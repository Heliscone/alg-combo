\documentclass[11pt]{scrartcl}
\usepackage{evan}

\title{Algebraic Combinatorics HW 3}
\author{Evan L, Dallin G, Alexander D, and Sawyer V}
\date{3-1-2024}
\definecolor{palegreen}{rgb}{0.6, 0.98, 0.6}
\begin{document}
\maketitle
\setcounter{section}{1}
\begin{problem}[\textcolor{red}{Symmetric polynomial and unimodality}; Extra-Credit]
    $f(x)= p_0 + p_1x + p_2x^2 + \cdots + p_nx^n$ is \textit{symmetric} if for all $i$, $$p_i=p_{n-i}$$ It is \textit{unimodal} if for some fixed $j$, $$p_0\le p_1\le \cdots \le p_{j-1}\le p_j \ge p_{j+1}\ge \cdots \ge p_{n-1}\ge p_n$$ Let $F(q)$, $G(q)$ be symmetric and unimodal polynomials with non-negative real coefficients. Show that $F(q)G(q)$ is also symmetric (easy) and unimodal (less easy).
\end{problem}
\begin{proof}
Let $F$ and $G$ be $n$-degree polynomials with $F(q)=\sum f_iq^i$ and $G(q)=\sum g_iq^i$. Let $H(q)=F(q)G(q)=\sum h_i q^i$ be the $2n$-degree polynomial that is the product of $F$ and $G$. Then, \[h_i=\sum_{a+b=i}f_ag_b=\sum_{a+b=i}f_{n-a}g_{n-b}=\sum_{c+d=2n-i}f_cg_d=h_{2n-i},\] so $F(q)G(q)$ is symmetric.

Since $F$ is unimodal, we can assign it center term $f_j$. $F$'s symmetry also allows us to write it as a sum of ``chunks'' $x^{j-c}+x^{j-c+1}+\cdots+x^{j+c-1}+x^{j+c}$  centered at $f_j$. We can do similarly for $G$, writing it as a sum of ``chunks'' $x^{l-c}+x^{l-c+1}+\cdots+x^{l+c}$ with maximal coefficient $q_l$. Then, the product of $F$ and $G$ is the sum of products of said chunks, and since 
\begin{align*}
    &(x^{j-c}+x^{j-c+1}+\cdots+x^{j+c-1}+x^{j+c})(x^{l-d}+x^{l-d+1}+\cdots+x^{l+d})\\
    &=x^{j+l-c-d}(1+x+\cdots+x^{2c})(1+x+\cdots+x^{2d})
\end{align*}
is unimodal, $F(q)G(q)$ is a sum of unimodal polynomials and thus unimodal.
\end{proof}
\begin{problem}[\textcolor{red}{Log-concavity of Binomial coefficients}]
    A sequence $a_1,a_2, a_3, \cdots, a_n$ is \textcolor{red}{\textit{logarithmically concave}} if $$a_i^2\ge a_{i-1}a_{i+1}~;~ \forall i$$

    \begin{enumerate}[(i)]
        \item Show that if a sequence of positive terms is log-concave, then it is also unimodal.\smallskip
        \item It is easy to see algebraically that the sequence $\binom{n}{0},\binom{n}{1}, \binom{n}{2}, \cdots, \binom{n}{n}$ is log-concave. Give a combinatorial proof of this fact.
    \end{enumerate}
\end{problem}
\begin{proof}
    \begin{enumerate}[(i)]
        \item We rewrite the given relation as 
        \[
            \frac{a_i}{a_{i-1}} \ge \frac{a_{i+1}}{a_i}.
        \]
        
        This means that the ratio of consecutive terms is non-increasing. Thus, if the sequence is not either nonincreasing or nondecreasing (in which case it is naturally unimodal), it is initially increasing and must decrease at some point. This means that the sequence is unimodal.
        \item \textit{Main Proof}: We show that $\frac{\binom{n}{k}}{\binom{n}{k-1}}\geq\frac{\binom{n}{k+1}}{\binom n k}$. The LHS is the ratio of the number of $k$-subsets of a set with $n$ elements to the number of $k-1$-subsets of the same set. There are $n-(k-1)$ ways to extend this subset to a $k$-subset, and each $k$-subset can be formed from $k$ different $k-1$-subsets. Thus, the LHS is $\frac{n-(k-1)}{k}$, and it follows that the RHS is $\frac{n-k}{k+1}$. $\frac{n-k+1}{k}\geq \frac{n-k}{k+1}$ holds when \[nk-k^2+k+n-k+1\geq kn-k^2\implies n\geq -1,\] as desired.

        \textit{Alternate Proof}: If we have $n$ distinct people in a room, there are $\binom nk ^2$ ways to give $k$ people red cards and $k$ people green cards. There are naturally $\binom n{k-1}\binom n{k+1}$ ways to give $k-1$ red cards and $k+1$ green cards. 

        Suppose we start by assigning $k-1$ green and red cards. Then, there are of course $(n-(k-1))^2$ ways to assign one more green and one more red card. In order to assign two more red cards, we have $\binom{n-(k-1)}{2}$ ways, noting that a person cannot have more than one card of each color. Clearly, \[(n-(k-1))^2\geq \frac{(n-(k-1))(n-(k-1)-1)}{2},\] so there are more ways to set up a $k$-$k$ scenario than a $k-1$-$k+1$ scenario, so $\binom n k ^2\geq \binom{n}{k-1}\binom{n}{k+1}$, as desired.
    \end{enumerate}
\end{proof}

\begin{problem}[\textcolor{red}{Uniqueness in Sperner's Thm}]
    Show that equality in Sperner's Theorem for $B_n$ is achieved only by the middle (middle two) rank(s) if $n$ is even (odd). (\textit{Hint}: If not, then move the example closer to the middle rank(s))
\end{problem}
\begin{proof}
    Let $l> \frac{n+1}{2}$ correspond to a minimum above-middle rank $l$. Consider the section of the graph of elements with ranks $l$ ($A$) and $l-1$ ($\delta A$). The shadow of $A$ is evidently $\delta A$, and we aim to show that there are more elements in $\delta A$ than in $A$. Considering the conections between $A$ and $\delta A$ gives us a bipartite graph, where each vertex of $A$ is connected to $l$ vertices in $\delta A$ ($l$ elements to remove) and each vertex of $\delta A$ is connected to $n-l+1$ vertices in $A$ ($n-(l-1)$ vertices not in any given element of $\delta A$). For $l$ as chosen, notice that \[\frac{l}{n-l+1}> 1,\] so there are more elements in $\delta A$ than $A$, and so equality cannot be achieved in both $A$ and $\delta A$ (the middle-rank). This argument also shows that the ranks above middle strictly increase in number as we approach the middle rank.

    A converse argument, with $l<\frac{n-1}{2}$ and $A$ and $\delta A$ constructed as before, gives the same bipartite graph, but $\frac{l}{n-l+1}<1$, so as we move down, the ranks decrease in size. 

    Thus, beacuse of this strict approach to equality, equality is only achieved when $\frac{n-1}{2}\leq l\leq \frac{n+1}{2}$, which is the middle (middle two) rank(s) if $n$ is even (odd), as desired.
\end{proof}

\begin{problem}[\textcolor{red}{A generalization of Sperner's Thm}]
    Let $P$ be a rank-symmetric, rank-unimodal poset. Show that if $P$ has a symmetric chain decomposition, then it has \textcolor{red}{\textit{strong Sperner property}}, that is, for any $j\ge 1$, the largest size of a union of $j$ antichains is equal to the size of the largest $j$ levels of $P$. (\textit{Remark}: $j=1$ corresponds to Sperner's Theorem when $P=B_n$)
\end{problem}

\begin{proof}


    Consider a collection $A$ of $j$ antichains $A_1,A_2,\dots,A_j$. The number of intersections between any chain $C_i$ in the symmetric decomposition $P=C_1\cup C_2\cup\cdots\cup C_p$ and $A$ is at most the minimum of $|C_i|$ and $j$, as no chain can intersect an antichain more than once. Also, since the $C_i$ are disjoint by definition, \[|F|\leq\sum_i |C_i\cap F|=\sum_i \min\{|C_i|,j\},\] and we can rewrite this sum based on the length of the chains in the symmetric chain decomposition.  
    
    We find the number of chains $C_i$ with $|C_i|=l$. By the chain's symmetry, the chain runs from the $\frac{n-(l-1)}{2}^{\text{th}}$ rank to the $\frac{n+l-1}{2}^{\text{th}}$ rank ($|C_i|=(l-1)+1=l$). The number of such chains is the number of chains that reach rank $\frac{n-l+1}{2}$ minus the number of chains that reach rank $\frac{n-l-1}{2}$, which is also $p_{\frac{n-l+1}{2}}-p_{\frac{n-l-1}{2}}$. For ease of calculation, we can define a permutation\footnote{this step was due to a hint from another person who was at camp with me.} $f(x)$ on $0,1,\dots,n+1$ to take care of the parity issues with $n-l-1$ while arranging them in decreasing order. That is, $f(x)$ gives the rank that is $\left\lfloor\frac{x}{2}\right\rfloor$ ranks away from the middle rank(s). Explicitly, for odd $n$, $f(x)=\frac{n+(-1)^x\left(2\left\lfloor\frac{x}2\right\rfloor+1\right)}2$; dual definition for even $n$. Thus, we can simplify our expression for the number of chains with $l$ vertices to $p_{f(l-1)}-p_{f(l)}$. When $l$ is an invalid parity, we have $f(l-1)=f(l)$, so we can write
    \begin{align*}
        \sum_i\min\{|C_i|,j\}&=\sum_{l=1}^{j-1}l|\{i:|C_i|=l\}|+\sum_{l=j}^{n+1}j|\{i:|C_i|=l\}|\\
        &=\sum_{l=1}^{j-1}l(p_{f(l-1)}-p_{f(l)})+j\sum_{l=j}^{n+1}p_{f(l-1)}-p_{f(l)}\\
        &=\sum_{l=1} ^{j}p_{f(l-1)}+0-p_{n+1}\\
        &=\sum_{l=0}^{j-1}p_{f(l)},
    \end{align*}
    which is precisely the sum of the $j$ largest $p_i$.
\end{proof}

\end{document}

\documentclass[11pt]{scrartcl}
\usepackage{evan}
\title{HW4}
\author{People}
\date{Date}
\definecolor{palegreen}{rgb}{0.6, 0.98, 0.6}
\begin{document}
\maketitle
\begin{problem}[\textcolor{red}{Some (counter)-example}]\phantom{0}

    \begin{enumerate}[(i)]
        \item Give an example of a finite graded poset $P$ with the Sperner property, together with a group $G$ acting on $P$, such that $P/G$ is \textit{not} Sperner.
        \item Consider the poset $P$ whose Hasse diagram is given by
    \end{enumerate}
\end{problem}
\begin{proof}
    \begin{enumerate}[(i)]
        \item We draw a Hasse diagram for $P$:
        \begin{center}
            \begin{asy}
                size(4cm);
                pair A=(0,0),B=(1,0),C=(2,0),D=(3,0),AA=(0,1),BB=(1,1),CC=(2,1),DD=(3,1);
                dot(A);dot(B);dot(C);dot(D);dot(AA);dot(BB);dot(CC);dot(DD);
                draw(A--AA);draw(A--BB);draw(A--CC);draw(A--DD);draw(B--BB);draw(B--CC);draw(B--DD);draw(C--CC);draw(C--DD);draw(D--DD);
                label("$1$",AA,N);
                label("$2$",BB,N);
                label("$3$",CC,N);
                label("$4$",DD,N);
                label("$5$",A,S);
                label("$6$",B,S);
                label("$7$",C,S);
                label("$8$",D,S);
            \end{asy}
        \end{center}
        We see that $P$ is Sperner by inspection; its largest antichain is of length four, and each rank has four elements. Let $G=(1,2)(5,6)$ be the group generated by the permutations $(1,2)$ and $(5,6)$, effectively collapsing $1,2$ and $5,6$ together, for a Hasse diagram of $P/G$:
        \begin{center}
            \begin{asy}
                size(4cm);
                pair B=(1,0),C=(2,0),D=(3,0),BB=(1,1),CC=(2,1),DD=(3,1);
                dot(B);dot(C);dot(D);dot(BB);dot(CC);dot(DD);
                draw(B--BB);draw(B--CC);draw(B--DD);draw(C--CC);draw(C--DD);draw(D--DD);
                label("$2$",BB,N);
                label("$3$",CC,N);
                label("$4$",DD,N);
                label("$6$",B,S);
                label("$7$",C,S);
                label("$8$",D,S);
            \end{asy}
        \end{center}
        \item Notice that $P$ is rank-symmetric, rank-unimodal, and Sperner. Also, $P$ is the Hasse diagram identical to the poset of nonisomorphic simple graphs with $7$ vertices. We define our subgroup $G$ to be the permutations $\hat\pi$ induced by automorphisms $\pi\in S_7$, where $\hat\pi\{i,j\}=\{\pi\cdot i,\pi\cdot j\}$.
    \end{enumerate}
\end{proof}
\begin{problem}
    THings and stuff.
\end{problem}
\begin{proof}
    $N_n$ is rank-symmetric because we have the same number of necklaces of weight $i$ and $n-i$, due to the bijection of simply flipping all digits. There are 
\end{proof}
\end{document}
\documentclass[11pt]{scrartcl}
\usepackage{evan}
\usepackage{asymptote}

\title{Algebraic Combinatorics HW 1}
\author{Evan Lim and Dallin Guisti}
\date{2-6-2024}
\definecolor{palegreen}{rgb}{0.6, 0.98, 0.6}
\begin{document}
\maketitle
\setcounter{section}{1}
\begin{problem}[\textcolor{red}{Closed Walks in $K_n$}]
    Find a combinatorial proof of the fact that $\#$ closed walks of length $l$ in $K_n$ from some vertex to itself is \[\frac{1}{n}\left((n-1)^l+(n-1)(-1)^l\right)\]
\end{problem}
\begin{proof}
    By symmetry, we can simply count the number of closed $l$-walks from $v_1$. We write each walk as an ordered list of vertices, so that we must find the number of walks 
    \[v_{i_1},v_{i_2},\cdots,v_{i_{l}},v_{i_{l+1}}\] with all $v_{i_k}\neq v_{i_{k+1}}$ and $v_{i_1}=v_{i_{l+1}}= v_1$. 
    
    We start by counting the number of walks if we allow $v_{i_l}=v_{i_{l+1}}$, and then subtract the overcounting. After the first, each of the $l-1$ subsequent vertices have $n-1$ options, for $(n-1)^{l-1}$ total sequences. We msut now subtract the number of sequences where $v_{i_{l-1}}=v_1$. 
    
    Conceptually, we are counting the number of closed $l$ walks by counting the number of open $l-1$ walks, which is the number of total $l-1$ walks minus the number of closed $l-1$-walks.\footnote{This suggests that we could complete via an inductive proof, using \[(n-1)^{l}-\frac{1}{n}\left((n-1)^{l-1}+(n-1)(-1)^{l-1}\right)=\frac{1}{n}\left((n-1)^{l}+(n-1)(-1)^l\right).\]}
    
    We now subtract the number of sequences where $v_{i_{l}}=v_1$, 
    Repeating the same reasoning, we can overcount by counting $(n-1)^{l-2}$ sequences, and then subtracting the number of sequences where $v_{i_{l-1}}=v_{i_l}=v_1$. We continue this process, finishing with $(n-1)^1$ closed $2$-walks, making the number of closed $l$-walks from a vertex to itself is
    \begin{align*}
        &(n-1)^{l-1}-(n-1)^{l-2}+(n-1)^{l-3}+\cdots+(-1)^l(n-1)\\
        &=\sum_{k=1}^{l-1}(n-1)^k(-1)^{l-1+k}\\
        &=(-1)^{l-1}\sum_{k=1}^{l-1}(1-n)^k\\
        &=(-1)^{l-1}\frac{(1-n)\left((1-n)^{l-1}-1\right)}{1-n-1}\\
        &=(-1)^{l}\frac{(1-n)^{l}-(1-n)}{n}\\
        &=\frac{1}{n}\left((n-1)^l+(n-1)(-1)^l\right),
    \end{align*}
    as desired.
\end{proof}
\begin{problem}[\textcolor{red}{Eigenvalues of some bipartite graphs}]
    \phantom{0}
    \begin{enumerate}[(i)]
        \item Let $G[A,B]$ be a bipartite graph with partite sets $A,B$. Show by a walk-counting argument that the non-zero eigenvalues of $G$ come in pairs $\pm \lambda$.\\
        (\textcolor{red}{Eigenvalues of $K_{rs}$}) Consider the complete bipartite graph $K_{r,s}$ (that is, having partite sets of size $r$ and $s$)
        \item Use purely combinatorial reasoning to compute the number of closed walks of length $l$ in $K_{r,s}$.
        \item Deduce the eigenvalues of $K_{r,s}$.\\
        (\textcolor{red}{Eigenvalues of $K_{n,n}-nK_2$}) Let $H_n$ be the graph $K_{n,n}$ with a perfect matching removed.
        \item Show that the eigenvalues of $H_n$ are \[\pm 1(n-1\text{ times}),\pm(n-1)\text{(once each)}.\]
    \end{enumerate}
\end{problem}
\begin{proof}
    \phantom{0}
    \begin{center}
        \begin{asy}
            size(3cm);
            draw(ellipse((-5,0),2,5));
            draw(ellipse((5,0),2,5));
            label("$A$",(-5,-7));
            label("$B$",(5,-7));
            pair A1,A2,A3,B1,B2,B3;
            A1 = (-5,3);
            A2 = (-5,0);
            A3 = (-5,-3);
            B1 = (5,3);
            B2 = (5,0);
            B3 = (5,-3);
            draw(A1--B2);
            draw(A2--B1);
            draw(A3--B3);
            dot(A1); dot(A2); dot(A3); dot(B1); dot(B2); dot(B3);
        \end{asy}
    \end{center}
    \begin{enumerate}[(i)]
        \item Every step on a walk takes us between partite sets $A$ and $B$. Thus, there are no $2l+1$-walks, meaning that \[\sum (\lambda_i)^{2l+1}=0,\] so \[\sum(-\lambda_i)^{2l+1}=\sum (\lambda_i)^{2l+1}=0.\] As \[\sum(\lambda_i)^{2l}=\sum(-\lambda_i)^{2l},\] $\sum\lambda_i^k$ and $\sum(-\lambda_i)^k$ agree for all positive integers $k$, so the $-\lambda_i$ are simply a permutation of the $\lambda_i$, meaning that all nonzero eigenvalues come in $\pm\lambda$ pairs.
        \begin{center}
            \begin{asy}
                size(3cm);
                draw(ellipse((-5,0),2,5));
                draw(ellipse((5,0),2,5));
                label("$r\phantom{0}(A)$",(-5,-7));
                label("$s\phantom{0}(B)$",(5,-7));
                pair A1,A2,A3,B1,B2,B3;
                A1 = (-5,3);
                A2 = (-5,0);
                A3 = (-5,-3);
                B1 = (5,3);
                B2 = (5,3);
                B3 = (5,-3);
                draw(A1--B1);draw(A1--B2);draw(A1--B3);
                draw(A2--B1);draw(A2--B2);draw(A2--B3);
                draw(A3--B3);draw(A3--B2);draw(A3--B3);
                dot(A1); dot(A2); dot(A3); dot(B1); dot(B2); dot(B3);
            \end{asy}
        \end{center}
        \item Call the partite with $r$ elements $A$ and the partite with $s$ elements $B$. If $l$ is odd, there are zero walks. So, we assume $l$ is even. If we begin our $l$-walk in $A$, we know go from $A$ to $B$ $\frac{l}2$ times and $B$ to $A$ $\frac{l}2$ times. Each time we go from $A$ to $B$, we have $s$ options. Each time we go from $B$ to $A$, we have $r$ options, except the last step, at which point we must return to our original vertex, for which we have $r$ choices. There are thus $s^{l/2}r^{l/2-1}r=(rs)^{l/2}$ $l$-walks beginning in $A$, and an identical argument gives $(rs)^{l/2}$ $l$-walks beginning in $B$. The number of $l$-walks is thus \[\begin{cases}0&l\text{ odd}\\2(rs)^{\frac{l}{2}}&l\text{ even}\end{cases}.\]
        \item $\sum\lambda_i^l$ and $(rs)^l+(-rs)^l+(n-2)\cdot(0)^l$ agree for all positive $l$, so the eigenvalues of $K_{r,s}$ are \[\pm rs,0\text{ }(r+s-2\text{ times}).\]
        We now consider the $K_{n,n}-nK_2$ graph, providing $n=3$ as an example:
        \begin{center}
            \begin{asy}
                size(3cm);
                draw(ellipse((-5,0),2,5));
                draw(ellipse((5,0),2,5));
                label("$A$",(-5,-7));
                label("$B$",(5,-7));
                pair A1,A2,A3,B1,B2,B3;
                A1 = (-5,3);
                A2 = (-5,0);
                A3 = (-5,-3);
                B1 = (5,3);
                B2 = (5,0);
                B3 = (5,-3);
                draw(A1--B1,dashed+red);draw(A1--B2);draw(A1--B3);
                draw(A2--B1);draw(A2--B2,dashed+red);draw(A2--B3);
                draw(A3--B1);draw(A3--B3,dashed+red);draw(A3--B2);
                dot(A1); dot(A2); dot(A3); dot(B1); dot(B2); dot(B3);
            \end{asy}
        \end{center}
        \item We aim to find the number of $l$-walks (for even $l$) on $K_{n,n}-nK_2$. If we write the partites as $a_1,a_2,\cdots, a_n$ and $b_1,b_2,\cdots,b_n$ such that the $a_i$ and $b_i$ are not connected, but $a_i$ and $b_j$ for $i\neq j$ are connected, then our walk-counting problem becomes analogous to the $K_n$ problem. Any valid will alternate between $A$ and $B$, but have \textit{no index repeated twice in a row}. That is, $a_1b_2$ is a valid step, while $a_1b_1$ is not. We can use this to establish an bijection between $l$-walks starting in $A$ and $l$-walks on $K_n$, meaning that the \textit{total} number of $l$-walks on $K_{n,n}-nK_2$ is \[2n\left(\frac{1}{n}\left((n-1)^l+(n-1)(-1)^l\right)\right)=2(n-1)^l+2(n-1)(-1)^l.\] For odd $l$, the number of walks is $0$, and so $\sum\lambda_i^l$ agrees for all positive $l$, as \[(n-1)^l+(1-n)^l+(n-1)((-1)^l+(1)^l),\] meaning that our eigenvalues are \[\pm 1(n-1\text{ times}),\pm(n-1)\text{(once each)}.\]
    \end{enumerate}
\end{proof}
\begin{problem}[\textcolor{red}{On the largest eigenvalue of $A(G)$}; \textbf{Extra credit}]\phantom{0}

\begin{enumerate}[(i)]
    \item Let $G$ be a graph with max degree $\Delta(G)$. Let $\lambda_1$ be the largest eigenvalue of $A(G)$. Show that $\lambda_1\leq\Delta(G)$.
    \item Let $G$ be a simple graph with $m$ edges. Show that $\lambda_1\leq \sqrt{2m}$.
\end{enumerate}
\end{problem}
\begin{proof}
    \begin{enumerate}[(i)]
        \item Let $u$ be the corresponding eigenvector to the eigenvalue $\lambda_1$. By computation of $A(G)u = \lambda_1 u$, we see $\sum_{i} {A_{ij}u_i} = \lambda_1 u_j$ for every index $j$ of $u$. Choose $j$ such that $u_j$ is maximal magnitude. It follows that
            \[|\lambda_1| |u_j| = \left|\sum_{i} {A_{ij}u_i}\right|=|u_i|\left|\sum_i A_{ij}\right|\leq |u_j||\deg(v_j)|,\] 
            so 
           \[\lambda_1\leq|\lambda_1| \leq|\deg(v_j)|\leq\Delta(G),\]
        as desired.
        \item Since $G$ is simple (no loops), its adjacency matrix is traceless, so the sum of its eigenvalues is zero. The number of two walks is both the sum of the squares of the eigenvalues and $2m$ (each edge corresponds to two $2$-walks), so \[\lambda_1^2=\left(-\sum_{i=2}^n1\cdot\lambda_i\right)^2\leq \left(\sum_{i=2}^n1^2\right)\left(\sum_{i=2}^n\lambda_i^2\right)=(n-1)(2m-\lambda_1^2).\] It follows that \[\lambda_1^2\leq \frac{(n-1)2m}{n}\leq 2m,\] so $\lambda_1\leq \sqrt{2m}$.
    \end{enumerate}
\end{proof}
\begin{problem}
\phantom{0}
\begin{enumerate}[(i)]
    \item Start with $n$ coins heads up. Choose a coin at random and turn it over. Do this a total of $m$ times. What is the probability that all coins will have heads up?
    \item Same as (i), except now compute the probability that all coins have tails up.
    \item Same as (i), but now we turn over two coins at a time.
\end{enumerate}
\end{problem}
\begin{proof}[Solution]
    \phantom{0}
\begin{enumerate}[(i.)]
    \item Flipping $m$ coins can be modeled as an $m$-walk on the graph of the $n$-hypercube, $Q_n$. Thus, the probability we seek is \[\frac{\#\text{closed }m-\text{walks}}{\#m-\text{walks}}\text{ on }\ZZ_2^n.\] There are $n^m$ such walks, both because each vertex has degree $n$/each step flips one of $n$ coordinates and that we choose amongst $n$ coins each flip. 
    
    In order to find the number of closed $m$-walks on $Q_n$, we consider the eigenvalues of $A(Q_n)$, which are of the form $n-2i$ as $i$ ranges from $0$ to $n$, where $\lambda_i=n-2i$ has multiplicity $\binom{n}{i}$. Thus, the number of closed $m$-walks on $Q_n$ is
    \begin{align*}
        \sum_{i = 0}^{n} \binom{n}{i} (n-2i)^m,
    \end{align*}
    so the desired probability is \[\dfrac{\sum_{i = 0}^{n} \binom{n}{i} (n-2i)^m}{n^m}.\]

    \item Starting with all heads and ending with all tails is akin to walking from $v_1$ to $v_{2^n}$. So, we seek the ratio \[\frac{((A(Q_n))^m)_{1,2^n}}{n^m},\] and applying Corollary 2.5 from the textbook gives a probability of \[\frac{\sum_{i=0}^n\sum_{j=0}^k(-1)^j\dbinom{k}{j}\dbinom{n-k}{i-j}(n-2i)^l}{2^nn^m}.\]
    \item We aim to find the adjacency matrix $A(G)$ of the graph represented by flipping two coins at a time. $A(Q_n)^2$ represents the number of ways to flip two coins, but it allows us to flip the same coin twice in a row, result in nonzero diagonals that we must eliminate. Each $(A(Q_n))^2_{ii}=n$ because we have $n$ coins that we can flip and then un-flip. One further consideration is that $A(Q_n)^2-nI_n$ eliminates the loops, but still overcounts by a factor of two (it regards flipping $A$ then $B$ as distinct from flipping $B$ then $A$). Thus, \[A(G)=\frac12(A(Q_n)^2-nI_n),\] from which we find that the eigenvalues of $A(G)$ are simply the eigenvalues of $A(Q_n)^2$ shifted by $n$, noting that
    \begin{align*}
        \det(A(Q_n)^2-\lambda_iI_n)&=0\\
        \det(A(Q_n)^2-nI_n-(\lambda_i-n)I_n)&=0\\
        \det\left(\frac12(A(Q_n)^2-nI_n)-\frac{\lambda_i-n}{2}I_n\right)&=0\\
        \det\left(A(G)-\frac{\lambda_i-n}{2}I_n\right)&=0.
    \end{align*}
    The eigenvalues of $A(Q_n)^2$ are $(n-2i)^2$ for $0\leq i\leq n$ with multiplicity $\binom n i$. So, the eigenvalues of $A(G)$ are \[\frac{(n-2i)^2-n}{2}\] as $i$ ranges from $0$ to $n$ with multiplicity $\binom n i$. The number of closed $m$-walks beginning at the zero vertex is then \[\frac{1}{n}\sum_{i=0}^n\binom n i\left(\frac{(n-2i)^2-n}{2}\right)^m,\] and since there are $\binom{n}{2}^m$ total $m$-walks beginning at the zero vertex, our probability is \[\frac{\sum_{i=0}^n\binom n i\left(\frac{(n-2i)^2-n}{2}\right)^m}{n\frac{(n^2-n)^m}{2^m}}=\frac{\sum_{i=0}^n\binom n i ((n-2i)^2-n)^m}{n(n^2-n)^m}.\]
\end{enumerate}
\end{proof}
\begin{problem}
    Let $G_n$ be the graph with vertex set $\ZZ_2^n$ with the edge set defined as: $u$ and $v$ are adjacent iff they differ in exactly two coordinates (that is, $\omega(u+v)=2$). What are the eigenvalues of $G_n$?
    
\end{problem}
\begin{proof}
    Recall from lecture that the eigenvalues of $A(Q_n)$ are of the form $n-2i$ as $i$ ranges from $0$ to $n$, where $\lambda_i=n-2i$ has multiplicity $\binom{n}{i}$. Consider $A(Q_n)^2$. We aim to show that \[A(Q_n)^2=nI_n+2A(G_n)\] via a walk counting argument. $A(Q_n)^2_{ij}$ counts the number of $2$-walks from $i$ to $j$ (vertices on the hypercube). When $i=j$, this is precisely $\deg i=n$. When $i\neq j$, for there to be a $2$-walk from $i$ to $j$, we must have $\omega(i+j)=2$. Each step flips one coordinate of $i$, and since there must be exactly two coordinates to flip, we have $2!=2$ such walks. Thus, $A(Q_n)^2_{ij}=2$ when $\omega(i+j)=2$. $2A(G_n)$ is the matrix with $2$s when $\omega(i+j)=2$ and zeroes elsewhere, and $nI_n$ is the matrix with $n$s along the diagonal. Therefore, \[A(Q_n)^2=nI_n+2A(G_n),\] so for any given eigenvalue $\lambda_i$ of $Q_n$, since $A(Q_n)$ is diagonalizable, $\lambda_i^2$ is an eigenvalue of $A(Q_n)^2$. Thus,
    \begin{align*}
        \det(A(Q_n)^2-\lambda_i^2I_n)&=0\\
        \det(nI_n+2A(G_n)-\lambda_i^2I_n)&=0\\
        \det(2A(G_n)-(\lambda_i^2-n)I_n)&=0\\
        \det\left(A(G_n)-\frac{\lambda_i^2-n}{2}I_n\right)&=0,
    \end{align*} 
    so $\frac{\lambda_i^2-n}{2}=\frac{(n-2i)^2-n}{2}$ is an eigenvalue of $G_n$, as $i$ ranges from $0$ to $n$.
\end{proof}
\end{document}
